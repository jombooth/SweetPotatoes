\documentclass[12pt, letterpaper, draft]{article}
\usepackage[top=1in, bottom=1in, left=1in, right=1in]{geometry}
\usepackage[T1]{fontenc}
\usepackage[utf8]{inputenc}
\usepackage{mathptmx}
\usepackage{lipsum}
\usepackage{apacite}

\begin{document}
\title{Sweet Potatoes}
\author{Jeff Rogers, Jo Booth, Alex Amis, Max Ubinas}
\date{\today}
\maketitle

\begin{abstract}
Abstract should contain
\begin{itemize}
    \item Statement of need: 3 sentences max. Brief overview of nutritional
    problem, target population, and goals of the intervention.
    \item Project description: summarize intervention design and implementation,
    and plan for evaluation/monitoring.
    \item Discussion: feasibility, sustainability, and further directions.
\end{itemize}
\end{abstract}

\section*{Statement of Need (2 pages)}
\begin{itemize}
    \item Nutritional problem?
    \item Prevalence/incidence of the problem?
    \item Causes and consequences of problem?
    \item Why is this important to nutrition and global health
    \item What population is at highest risk?
    \item Which population is being targeted in this intervention and why?
\end{itemize}

\section*{Program Aims and Objectives (a few sentences)}
\begin{itemize}
    \item What is the primary outcome of the intervention?
\end{itemize}

\section*{Description of Project (4-5 pages)}
\subsection*{Intervention Design}
\begin{itemize}
    \item Key components of intervention? Specific strategies?
    \item Strength of evidence in support of this intervention?
    \item Why is this the best intervention? Are there alternatives?
\end{itemize}
\subsection*{Implementation Strategy}
\begin{itemize}
    \item How will this be implemented? Include timeline, implementing agencies/partners,
    necessary prerequisites and supplies, and plan for reaching the target population.
\end{itemize}
\subsection*{Data collection, Data Analysis, Monitoring and Evaluation}
\begin{itemize}
    \item How will data be collected and analyzed?
    \item Anticipated impact of intervention on the nutritional problem?
    \item How will the impact of the intervention be measured? Include specific
    indicators, and how they will be measured, including assessment method and frequency.
\end{itemize}
\section*{Discussion (1-2 pages)}
\subsection*{Feasibility}
\begin{itemize}
    \item Why are you convinced that you'll be able to implement this program?
    \item What factors are essential for the success of this intervention?
    \item Is this intervention sustainable and does it integrate well
    with related activities that are ongoing in the intervention area?
\end{itemize}
\subsection*{Strengths and Limitations}
\begin{itemize}
    \item Discuss the strengths and limitations of this intervention
    \item Are there other interventions you would also recommend?
\end{itemize}
\subsection*{Implications and Future Directions}
\begin{itemize}
    \item What do the results of this intervention mean for future programmatic,
    research, policy, and clinical interventions?
\end{itemize}

\section*{Conclusion (One paragraph)}
Summarize key points

\section*{Acknowledgements (Not included in page count)}
One paragraph describing individual contributions of each author


\bibliographystyle{apacite}
\bibliography{main.bib}
\end{document}
