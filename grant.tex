\documentclass[12pt, letterpaper, draft]{article}
\usepackage[top=1in, bottom=1in, left=1in, right=1in]{geometry}
\usepackage[T1]{fontenc}
\usepackage[utf8]{inputenc}
\usepackage{mathptmx}
\usepackage{lipsum}
\usepackage{apacite}
\usepackage{setspace}
\usepackage{indentfirst} % indent the first paragraph in each section.

\doublespacing % double space

\begin{document}
\title{Sweet Potatoes}
\author{Jeff Rogers, Jo Booth, Alex Amis, Max Ubinas}
\date{\today}
\maketitle

\thispagestyle{empty}
\begin{abstract}
Abstract should contain
\begin{itemize}
    \item Statement of need: 3 sentences max. Brief overview of nutritional
    problem, target population, and goals of the intervention.
    \item Project description: summarize intervention design and implementation,
    and plan for evaluation/monitoring.
    \item Discussion: feasibility, sustainability, and further directions.
\end{itemize}
\end{abstract}
\pagebreak
\setcounter{page}{1}

\section{Statement of Need (2 pages)}
\begin{itemize}
    \item Nutritional problem?
    \item Prevalence/incidence of the problem?
    \item Causes and consequences of problem?
    \item Why is this important to nutrition and global health
    \item What population is at highest risk?
    \item Which population is being targeted in this intervention and why?
\end{itemize}

\section{Program Aims and Objectives} % (a few sentences)
% \begin{itemize}
%     \item What is the primary outcome of the intervention?
% \end{itemize}

Our intervention aims to reduce the incidence of hypovitaminosis A in what is today known as the Somali Region of Ethiopia. A 1993 paper \cite{wolde1993severe} showed that in a specific village in this region\footnote{This geographic region would then have been considered part of the Hararge Region of Ethiopia}, biochemical markers for Vitamin A were found to be dangerously low in the villagers, and clinical signs of hypovitaminosis A were present in many of the villagers. Although Ethiopia has now adopted a national supplementation program for vitamin A \cite{semba2008coverage}, hypovitaminosis A persists in the country, especially in rural regions \cite{demissie2010magnitude}. Our aim is to target individual rural villages, eliminating Vitamin A deficiency over the short term by providing supplement pills and subsequently over the long term by transitioning a fraction of the cereal-heavy local agriculture toward the growing of sweet potatoes, a crop naturally rich in beta-carotene and proven viable in Ethiopia \cite{belehu2003agronomical}. We are focused on tracking the effects of providing villagers with vitamin A over time, and will therefore longitudinally observe the rates of clinical signs of vitamin A deficiency (night blindness, corneal ulceration, and Bitot's spots) in tandem with the rates of disorders connected with malnutrition (in particular, stunting and wasting). Since vitamin A supplementation provides people with immunological benefits, we expect to see a decline in overall malnutrition in conjunction with the elimination of vitamin A deficiency in the villages we target.

\section{Description of Project (4-5 pages)}
\subsection{Intervention Design}
\begin{itemize}
    \item Key components of intervention? Specific strategies?
    \item Strength of evidence in support of this intervention?
    \item Why is this the best intervention? Are there alternatives?
\end{itemize}
\subsection{Implementation Strategy}
\begin{itemize}
    \item How will this be implemented? Include timeline, implementing agencies/partners,
    necessary prerequisites and supplies, and plan for reaching the target population.
\end{itemize}
\subsection{Data collection, Data Analysis, Monitoring and Evaluation}
\begin{itemize}
    \item How will data be collected and analyzed?
    \item Anticipated impact of intervention on the nutritional problem?
    \item How will the impact of the intervention be measured? Include specific
    indicators, and how they will be measured, including assessment method and frequency.
\end{itemize}
\section{Discussion (1-2 pages)}
\subsection{Feasibility}
\begin{itemize}
    \item Why are you convinced that you'll be able to implement this program?
    \item What factors are essential for the success of this intervention?
    \item Is this intervention sustainable and does it integrate well
    with related activities that are ongoing in the intervention area?
\end{itemize}
\subsection{Strengths and Limitations}
\begin{itemize}
    \item Discuss the strengths and limitations of this intervention
    \item Are there other interventions you would also recommend?
\end{itemize}
\subsection{Implications and Future Directions}
\begin{itemize}
    \item What do the results of this intervention mean for future programmatic,
    research, policy, and clinical interventions?
\end{itemize}

\section{Conclusion (One paragraph)}
Summarize key points
\pagebreak

\section{Acknowledgements (Not included in page count)}
One paragraph describing individual contributions of each author
\pagebreak
\pagestyle{empty}

\bibliographystyle{apacite}
\bibliography{main.bib}
\end{document}
