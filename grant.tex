\documentclass[12pt, letterpaper, draft]{article}
\usepackage[top=1in, bottom=1in, left=1in, right=1in]{geometry}
\usepackage[T1]{fontenc}
\usepackage[utf8]{inputenc}
\usepackage{mathptmx}
\usepackage{lipsum}
\usepackage{apacite} % APA style citations
\usepackage{setspace}
\usepackage{indentfirst} % indent the first paragraph in each section.



\begin{document}
\title{Combating Hypovitaminosis A in Rural Ethiopia}
\author{Jeff Rogers, Jo Booth, Alex Amis, Max Ubinas}
\date{\today}
\maketitle

\thispagestyle{empty}
% if you are planning to add a citation later
% write [CITATION] in the spot in the text you
% intend to add a citation to

% comment out items in each enumerate. if no
% items are left, comment out the whole enumerate
% to avoid compile errors.
\begin{abstract}
A 1993 study by Wolde-Gebriel, Gebru, Fisseha et al. illustrated that rural populations in the Hararge region of Ethiopia --- since repartitioned largely into today's Somali region of Ethiopia --- suffered from severe deficiencies of Vitamin A, with clinical signs of hypovitaminosis A observed in 116 of of 240 children within a single rural village. Though the Ethiopian government has since instituted a national Vitamin A supplementation program, much of the rural populace of Ethiopia remains unreached. Our goal is to identify a small, relatively unreached village in that region and to reduce hypovitaminosis A within that village by modifying the diet there and educating the villagers about the importance of Vitamin A in their diet.

To complement this national supplementation program and help reach rural communities, we propose a diet-based intervention in which we introduce the Vitamin A-rich sweet potato into the rural Somali regional diet. In a Conditional Cash Transfer, we will distribute sweet potato bare roots along with a small cash payment from a central location in the village. We will audit each recipient on a monthly basis to ensure that they are actually growing the crop and redistribute the cash payment based on compliance. We will, additionally, work with village leaders to organize educational sessions wherein the importance of Vitamin A is emphasized and the agriculture of sweet potatoes is taught. In order to evaluate the effectiveness of the intervention, we will longitudinally observe incidence rates of typical clinical markers of Vitamin A deficiency as well as overall mortality in the village we expose to our intervention, and compare this data to measurements taken in a similar but unexposed village nearby. 

Our project is feasible to implement in the region because sweet potatoes will grow well in the area's dry soil and the resultant crop can easily be added to recipes for traditional \emph{injera} and \emph{wat} pastes. The introduction of a new crop, rather than additional supplementation, allows the intervention to be sustained through generations. If the results of the intervention are positive then we propose a future direction of making it into an ``Adopt-a-Village'' program, wherein every \$10 million in contributions provides us with the resources to conduct an identical intervention in another village in the region.
\end{abstract}

\pagebreak
\setcounter{page}{1}

\doublespacing % double space

\section{Statement of Need} % (2 pages)
% \begin{itemize}
%     % \item Nutritional problem?
%     % \item Prevalence/incidence of the problem?
%     % \item Causes and consequences of problem?
%     % \item Why is this important to nutrition and global health
%     % \item What population is at highest risk?
%     % \item Which population is being targeted in this intervention and why?
% \end{itemize}

The nutritional problem we intend to target is primarily the deficiency of Vitamin A. This deficiency poses a number of related problems for a person's overall nutritional status - while the clinically observable effects of hypovitaminosis A include Bitot's spots, xeropthalmia, corneal ulceration, and night blindness, the vitamin's important role in a healthy individual's overall immunological function entails that deficiency of this vitamin results in overall susceptibility to infection, which in turn leads to worsened overall nutritional status \cite{underwood1978hypovitaminosis}. Thus, although our intervention turns on the targeting of a specific micronutrient deficiency, we are optimistic that reducing the incidence of this deficiency will result in general improvements to the nutritional status of the people in the villages we target.

Numerous studies have established that Vitamin A deficiency is exceedingly common throughout Ethiopia \cite{demissie2010magnitude}. Indeed, the WHO considers a country to have an unacceptably high rate of Vitamin A deficiency if Bitot's spots are seen in more than 0.5\% of the populace, but a 1981 survey found a 1.0\% incidence of the Bitot's spots in Ethiopia overall, indicating that people in the country are at an extremely high risk of hypovitaminosis A. 

Further, this deficiency is at its worst in rural regions of the country \cite{haidar1999malnutrition} that are less reachable via the conventional distribution channels in Ethiopia's national infrastructure. We argue that although the overall risk of Vitamin A deficiency is high throughout Ethiopia, rural populations are most at risk for hypovitaminosis A. There are staggeringly high rates of the disease within specific villages in the Ethiopian countryside; to put the threat of the deficiency in perspective, a 1991 study conducted in a relatively remote village in the Hararghe region in fact found clinical signs of Vitamin A deficiency in 116 out of the 240 children in the village \cite{wolde1993severe}; this translates directly to a very high incidence of child mortality as well as child blindness. 

The cause of Vitamin A deficiency in Ethiopia is straightforward - vegetarian diets are common in the country, as poverty makes meat products inaccessible to many people. Traditional cuisine particular to the Ethiopian highlands revolves largely around the consumption of wheat \cite{gelalcha2000milling}, a grain that provides very little Vitamin A. Since the diet typical of people in the region is so poor in Vitamin A, to avoid the health effects the deficiency entails it is essential that an novel source of the vitamin be introduced.

Vitamin A deficiency entails a variety of very serious health concerns. The effects of hypovitaminosis A are broad and multifaceted, but generally they are centered about the simple fact that Vitamin A is necessary for a healthy immune system. The consequences of the deficiency going untreated include a 34\% greater risk of overall child mortality as compared with children who have received a Vitamin A supplement \cite{sommer1992vitamin}. Additionally, Vitamin A deficiency is one of the world's principal causes of childhood blindness \cite{world2009global}, a condition associated with poor overall quality of life - and furthermore, these cases of childhood blindness are completely preventable simply by providing the affected children with a source of Vitamin A. Yet another complication associated with Vitamin A deficiency is significantly increased risk of mother-to-child transmission (MTCT) of HIV \cite{semba1994maternal}; we argue that this threat should be taken especially seriously in the context of rural Ethiopia, where medical care is essentially nonexistent. 

With a long-term aim of combating the incidence of hypovitaminosis A in rural Ethiopia, we wish to target the members of a specific small rural community in order to explore the effects that modifying their diet to include significant amounts beta-carotene will have on their incidence of clinical signs of Vitamin A deficiency, as well as their mortality. It is our hope that the human benefits that our study will provide will change and improve lives - but in addition, we argue that the data we will collect about the effectiveness of the intervention we stage may pave the way for broader initiatives for eradicating Vitamin A deficiency from the Ethiopian countryside.

\section{Program Aims and Objectives} % (a few sentences)
% \begin{itemize}
%     \item What is the primary outcome of the intervention?
% \end{itemize}

Our intervention aims to reduce the incidence of hypovitaminosis A in what is today known as the Somali Region of Ethiopia. A 1993 paper \cite{wolde1993severe} showed that in a specific village in this region\footnote{This geographic region would then have been considered part of the Hararge Region of Ethiopia.}, biochemical markers for Vitamin A were found to be dangerously low in the villagers, and clinical signs of hypovitaminosis A were present in many of the villagers. Although Ethiopia has now adopted a national supplementation program for Vitamin A \cite{semba2008coverage}, hypovitaminosis A persists in the country, especially in rural regions \cite{demissie2010magnitude}. Our aim is to target individual rural villages, eliminating Vitamin A deficiency over the short term by providing supplement pills and subsequently over the long term by transitioning a fraction of the cereal-heavy local agriculture toward the growing of sweet potatoes, a crop naturally rich in beta-carotene and proven viable in Ethiopia \cite{belehu2003agronomical}. 

We are focused on tracking the effects of providing villagers with Vitamin A over time, and will therefore longitudinally observe the rates of clinical signs of Vitamin A deficiency (night blindness, corneal ulceration, and Bitot's spots) in tandem with the rates of disorders connected with malnutrition (in particular, stunting and wasting). Since Vitamin A supplementation provides people with immunological benefits, we expect to see a decline in overall malnutrition in conjunction with the elimination of Vitamin A deficiency in the villages we target.

\section{Description of Project (4-5 pages)}
\subsection{Intervention Design}
\begin{itemize}
    \item Key components of intervention? Specific strategies?
    \item Strength of evidence in support of this intervention?
    \item Why is this the best intervention? Are there alternatives?
\end{itemize}
\subsection{Implementation Strategy}
\begin{itemize}
    \item How will this be implemented? Include timeline, implementing agencies/partners,
    necessary prerequisites and supplies, and plan for reaching the target population.
\end{itemize}
\subsection{Data collection, Data Analysis, Monitoring and Evaluation}
\begin{itemize}
    \item How will data be collected and analyzed?
    \item Anticipated impact of intervention on the nutritional problem?
    \item How will the impact of the intervention be measured? Include specific
    indicators, and how they will be measured, including assessment method and frequency.
\end{itemize}
\section{Discussion (1-2 pages)}
\subsection{Feasibility}
\begin{itemize}
    \item Why are you convinced that you'll be able to implement this program?
    \item What factors are essential for the success of this intervention?
    \item Is this intervention sustainable and does it integrate well
    with related activities that are ongoing in the intervention area?
\end{itemize}

We argue that this program is feasible primarily because the material costs of the intervention are very low, as are the costs of the evaluations that we intend to conduct. Since we are evaluating the success of the intervention by observing for clinical signs of Vitamin A deficiency longitudinally and by monitoring all-cause mortality in our targeted village as well as a control village, we do not require any specialized equipment or highly trained personnel to conduct our evaluation. Similarly, the only materials we are providing are sweet potato bare roots, which are extremely inexpensive as the crop is very straightforward to grow and reproduce in Ethiopia, in addition to small sums of money to be used in our Conditional Cash Transfers. Since our Conditional Cash Transfers are only being used to offset profits lost to villagers in growing the low-nutrition grains they would typically grow for subsistence (in particular, teff and other cereals), very little money will be required to conduct these transfers, even over long periods of time. Additionally, because we will be targeting a relatively small population for our intervention (< 300 people), although we will be providing these same resources over a 10-year period, the resultant expenses will be very low.

Factors that are essential for the success of our intervention include: 
{\begin{itemize}
\setlength\itemsep{-1em}
\item Cooperative (but not necessarily educated or trained) local staff
\item Willingness of community members to engage with our education program
\item Acceptance of our offer of CCTs
\item Diligence of our CCT auditors in ensuring the sweet potatoes are grown
\item Acceptance of the new food items we will provide
\end{itemize}}

We expect that our biggest difficulties may come in finding locals who are connected to nearby towns or cities who are willing to help us coordinate with village leaders in order to educate them about growing sweet potatoes as well as about the importance of Vitamin A in human nutrition. We expect, however, that this challenge is primarily a financial one, and that we will have the resources to secure the manpower we need.

We have conducted research into the local diet in the region, and found that \emph{injera} and \emph{wat} make up much of the staple diet of people in the Somali region. The former item, \emph{injera}, is bread made from lightly fermented locally grown cereals, while the latter item \emph{wat} is a group of vegetable-based pastes that are eaten with the bread. Since sweet potatoes lend themselves to making flavorful pastes readily, we argue that introducing the sweet potatoes into the diet of people in the Somali region of Ethiopia will be very straightforward.




\subsection{Strengths and Limitations}

A core strength of our intervention is that it is designed to cover cases of hypovitaminosis A that will almost surely be missed by Ethiopia's national Vitamin A supplementation program. This program has had promising results in reducing the incidence of hypovitaminosis A in Ethiopia. In particular, incidence of xeropthalmia dropped from 10.9\% prior to the intervention to 4.5\% \cite{demissie2007process} after it. However, the distribution of the supplements it provides is dependent on the presence of modern infrastructure, which is unfortunately largely absent in much of the Ethiopian countryside. Our intervention attacks the problem of hypovitaminosis A in rural Ethiopia at the local level, and is designed to provide a long-term-sustainable dietary source of the vitamin in addition to knowledge of its importance. In addition, we have verified that there are favorable conditions for growing the sweet potatoes we plan to introduce in the Somali region of Ethiopia, so the local agricultural practices will need little revision in order to support the new crops.   

One of the significant limitations of our intervention is that its scope is necessarily limited. It is unlikely, for example, that implementing a program of this sort in every locale inaccessible by Ethiopia's infrastructure would be possible, due to the sheer numerosity of the small villages we are intending to target. While we have proposed to implement more programs like this one in other villages if our evaluation yields that our program is particularly successful, it is unlikely that interventions of this sort will ever fully address the problem of Vitamin A deficiency in rural Ethiopia. However, we are interested in tracking the effectiveness of diet modification in conjunction with educating villagers about the importance of Vitamin A, and we submit that in addition to the human benefits it provides that the data collected in our study might support the development of future programs aimed to improve the nutritional status of peoples living in rural Ethiopia.

Another intervention that we might recommend for combating hypovitaminosis A in the region over the shorter term would be an effort to distribute Vitamin A supplements to the villagers immediately upon arrival, and to gradually phase them out as the practice of growing and eating sweet potatoes becomes more commonplace and entrenched. Since we are focused on gathering information about the overall effectiveness of dietary modification in the region, however, we are not presently including this element in our intervention.

\subsection{Implications and Future Directions}
\begin{itemize}
    \item What do the results of this intervention mean for future programmatic,
    research, policy, and clinical interventions?
\end{itemize}

\section{Conclusion (One paragraph)}
Summarize key points
\pagebreak

\section{Acknowledgements (Not included in page count)}
One paragraph describing individual contributions of each author
\pagebreak
\pagestyle{empty}

\bibliographystyle{apacite}
\bibliography{main.bib}
\end{document}
