\documentclass[12pt, letterpaper, draft]{article}
\usepackage[top=-.2in, bottom=1in, left=1in, right=1in]{geometry}
\usepackage[T1]{fontenc}
\usepackage[utf8]{inputenc}
\usepackage{mathptmx}
\usepackage{setspace}
\usepackage{indentfirst} % indent the first paragraph in each section.

\doublespacing

\begin{document}
\title{Sweet Potatoes\\ \small{Jeff Rogers, Jo Booth, Alex Amis, Max Ubinas}}
\author{}
\date{\vspace{-13ex}}
\maketitle

\thispagestyle{empty}
\pagestyle{empty}
%Abstract should contain
% \begin{itemize}
%     %\item Statement of need: 3 sentences max. Brief overview of nutritional
%   % problem, target population, and goals of the intervention.
%     %\item Project description: summarize intervention design and implementation,
%     %and plan for evaluation/monitoring.
%     \item Discussion: feasibility, sustainability, and further directions.
% \end{itemize}

A 1993 study by Wolde-Gebriel, Gebru, Fisseha et al. illustrated that rural populations in the Hararge region of Ethiopia --- since repartitioned largely into today's Somali region of Ethiopia --- suffered from severe deficiencies of Vitamin A, with clinical signs of hypovitaminosis A observed in 116 of of 240 children within a single rural village. Though the Ethiopian government has since instituted a national Vitamin A supplementation program, much of the rural populace of Ethiopia remains unreached. Our goal is to identify a small, relatively unreached village in that region and to reduce hypovitaminosis A within that village by modifying the diet there and educating the villagers about the importance of Vitamin A in their diet.

To complement this national supplementation program and help reach rural communities, we propose a diet-based intervention in which we introduce the Vitamin A-rich sweet potato into the rural Somali regional diet. In a Conditional Cash Transfer, we will distribute sweet potato seedlings along with a small cash payment from a central location in the village. We will audit each recipient on a monthly basis to ensure that they are actually growing the crop and redistribute the cash payment based on compliance. We will, additionally, work with village leaders to organize educational sessions wherein the importance of Vitamin A is emphasized and the agriculture of sweet potatoes is taught. In order to evaluate the effectiveness of the intervention, we will longitudinally observe incidence rates of typical clinical markers of Vitamin A deficiency as well as overall mortality in the village we expose to our intervention, and compare this data to measurements taken in a similar but unexposed village nearby. 

Our project is feasible to implement in the region because sweet potatoes will grow well in the area's dry soil and the resultant crop can easily be added to recipes for traditional \emph{injera} and \emph{wat} pastes. The introduction of a new crop, rather than additional supplementation, allows the intervention to be sustained through generations. If the results of the intervention are positive then we propose a future direction of making it into an ``Adopt-a-Village'' program, wherein every \$10 million in contributions provides us with the resources to conduct an identical intervention in another village in the region.




\end{document}
